\mainmatter
\chapter{Εισαγωγή}

\noindent Τα προβλήματα εντοπισμού τομής \begin{english}(intersection detection)\end{english} μεταξύ γεωμετρικών σχημάτων (στο επίπεδο) ή στερεών (στον χώρο) είναι μια ευρεία κατηγορία υπολογιστικών προβλημάτων με μεγάλο πλήθος εφαρμογών. Γενικά ορισμένο, ένα πρόβλημα εντοπισμού τομής μεταξύ δύο γεωμετρικών αντικειμένων περιγράφεται ως εξής:

\emph{«Με δεδομένα τα γεωμετρικά χαρακτηριστικά δύο αντικειμένων και των συντεταγμένων που ορίζουν την θέση τους, να βρεθεί
αν τα αντικείμενα τέμνονται. Αν τέμνονται, να προσδιοριστούν τα γεωμετρικά χαρακτηριστικά της τομής.»}

Η ύπαρξη αλγοριθμικών λύσεων υψηλής απόδοσης για τα προβλήματα εντοπισμού τομής είναι θεμελιώδους σημασίας για έναν μεγάλο αριθμό εφαρμογών 
σε έναν μεγάλο αριθμό διαφορετικών πεδίων. Παραδείγματα πεδίων στα οποία χρησιμοποιούνται οι λύσεις αυτές είναι, μεταξύ άλλων, τα γραφικά υπολογιστών (και ιδιαίτερα το animation), τα προγράμματα σχεδιασμού υποβοηθούμενου από υπολογιστή (CAD), τα συστήματα πλοήγησης, τα γεωγραφικά πληροφοριακά συστήματα (GIS) καθώς και ένα ευρύ φάσμα επιστημονικών
και μηχανικών εξομοιωτών.

Η εργασία αυτή θα επικεντρωθεί συγκεκριμένα στο πρόβλημα τομής ευθείας-τετραέδρου. Το συγκεκριμένο  πρόβλημα έχει ιδιαίτερη σημασία καθώς τα τελευταία χρόνια η χρήση τεραεδρικών πλεγμάτων (tetrahedral meshes) έχει εφαρμοστεί με επιτυχία για την αναπαράσταση περίπλοκων τριδιάστατων όγκων σε ένα μεγάλο εύρος εφαρμογών.      

Για το πρόβλημα τομής ευθείας-τετραέδρου έχουν προταθεί διάφοροι αλγόριθμοι με διαφορετικά επίπεδα απόδοσης. Στα πλαίσια της εργασίας αυτής χρησιμοποιείται ο αλγόριθμος που παρουσιάζεται στην δημοσίευση \cite{PlatisTheoharis03}, καθώς και μια τροποποίησή του που προτάθηκε από τον C. Ericson \cite{ericson2005real}\cite{ericson2007blog}.

Σκοπός της εργασίας αυτής είναι η ανάπτυξη μιας εναλλακτικής υλοποίησης των αλγορίθμων αυτών η οποία προορίζεται για εκτέλεση από μονάδες επεξεργασίας γραφικών (GPU). Η χρήση των GPU για την ταχεία εκτέλεση υπολογισμών γενικού σκοπού (όχι, δηλαδή, υπολογισμών που αφορούν άμεσα τον σχηματισμό εικόνας) γνωρίζει ραγδαία αύξηση τα τελευταία χρόνια. Η αρχιτεκτονική των σύγχρονων GPU, η οποία βασίζεται στο μοντέλο παραλληλίας SIMD (Single Instruction Multiple Data), επιτρέπει μια σημαντικότατη αύξηση επιδόσεων σε ορισμένες εφαρμογές σε σύγκριση τις επιδόσεις τους αν εκτελούνταν στην CPU. Όπως θα γίνει εμφανές στην συνέχεια, οι αλγόριθμοι τομής ευθείας-τετραέδρου που αναφέρθηκαν μπορούν να προσαρμοστούν σε τέτοια μορφή ώστε να εκμεταλλευτούν τα πλεονεκτήματα της αρχιτεκτονικής των GPU και να επιτύχουν αυξημένες επιδόσεις.   

Το κείμενο της εργασίας διαρθρώνεται ως εξής:
Το κεφάλαιο 2 περιέχει τον τυπικό ορισμό του προβλήματος τομής ευθείας-τετραέδρου και την παρουσίαση των αλγορίθμων που χρησιμοποιούνται, καθώς και των βελτιστοποιήσεων που εφαρμόζονται στους αλγορίθμους αυτούς.
Στο κεφάλαιο 3 παρουσιάζεται μια επισκόπηση των μεθόδων προγραμματισμού εφαρμογών προορισμένων προς εκτέλεση σε GPU. Αναλύεται επιγραμματικά η αρχιτεκτονική των σύγχρονων GPU, παρουσιάζονται τα σύγχρονα περιβάλλοντα προγραμματισμού και η ιστορία τους και εξηγείται ο τρόπος προσαρμογής του προβλήματος τομής ευθείας-τετραέδρου στην επίλυση σε GPU. 
Στο κεφάλαιο 4 αναλύεται ο τρόπος υλοποίησης του λογισμικού της εργασίας. Παρουσιάζεται αρχικά η ακολουθιακή (εκτελούμενη σε CPU) εκδοχή του προγράμματος και ο προϋπάρχων κώδικας στον οποίο βασίστηκε. Στην συνέχεια αναφέρεται σε γενικές γραμμές ο σχεδιασμός και η δομή της εκδοχής που προορίζεται προς εκτέλεση σε GPU. Επίσης δίνονται οδηγίες χρήσης των εκτελέσιμων της εργασίας.
Στο κεφάλαιο 5 περιγράφεται η μεθοδολογία μέτρησης της απόδοσης των διάφορων μορφών του λογισμικού της εργασίας και παρατίθεται η ερμηνεία των αποτελεσμάτων που προέκυψαν με επιλεγμένα παραδείγματα.
Τέλος, το παράρτημα Α περιέχει το σύνολο των αποτελεσμάτων που προέκυψαν απο τις μετρήσεις απόδοσης. 

