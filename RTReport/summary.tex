\chapter*{Περίληψη}
\addcontentsline{toc}{chapter}{Περίληψη}
\pagestyle{plain}

\noindent Ο έλεγχος τομής ευθείας-τετραέδρου ανήκει σε μια κατηγορία προβλημάτων κεφαλαιώδους σημασίας για ένα μεγάλο αριθμό εφαρμογών στο πεδίο  των τριδιάστατων γραφικών. Παράλληλα, η πρόσφατη τάση προς την χρήση των μονάδων επεξεργασίας γραφικών (GPU) ως επεξεργαστών γενικού σκοπού έχει αποφέρει σημαντική αύξηση επιδόσεων σε ένα μεγάλο εύρος υπολογιστικά εντατικών εφαρμογών. Αντικείμενο της εργασίας αυτής είναι η μελέτη της εφαρμοσιμότητας των σύγχρονων τεχνικών προγραμματισμού GPU στους διαθέσιμους αλγορίθμους τομής ευθείας-τετραέδρου καθώς και η σύγκριση των επιδόσεων της προσέγγισης αυτής με τις παραδοσιακές υλοποιήσεις. Στα πλαίσια της εργασίας αυτής αναπτύχθηκε μια υλοποίηση των ταχύτερων αλγορίθμων τομής ευθείας-τετραέδρου σε μορφή κατάλληλη για εκτέλεση σε GPU.

Η παρούσα εργασία περιέχει τον τυπικό ορισμό του προβλήματος, την παρουσίαση των αλγορίθμων που χρησιμοποιούνται, την περιγραφή της υλοποίησης και τα συγκριτικά στοιχεία των επιδόσεων. Επίσης παρέχεται μια επισκόπηση των αρχιτεκτονικών των GPU και των τεχνικών προγραμματισμού τους.

\subsubsection*{Λέξεις-Κλειδιά}

\noindent Επεξεργασία σε GPU, Υπολογιστική Γεωμετρία, Έλεγχος τομής, Μαζική Παραλληλία, OpenCL, Ευθεία, Τετράεδρο, Παραλληλία SIMD 

\chapter*{Abstract}
\addcontentsline{toc}{chapter}{Summary}
\begin{english}
\noindent Ray-Tetrahedron intersection testing belongs to a class of computational problems that is fundamental to a wide range of applications in the field of 3D computer graphics. Additionally, the recent trend towards using Graphical Processing Units (GPUs) for general purpose computing has resulted in a significant performance increase in a wide number of computationally intensive applications. This project aims to determine the applicability of current GPU programming techniques to available ray-tetrahedron intersection algorithms and to compare the performance characteristics of this approach to traditional implementations. For the purposes of this project a GPU-targeted implementation of the fastest ray-tetrahedron intersection algorithms available has been developed. 

This report contains a formal definition of the ray-tetrahedron instersection problem, a presentation of the algorithms used, a description of the resulting implementation and data concerning its performance. Also provided is an overview of current GPU architectures and programming techniques.
\end{english} 

\subsubsection*{Keywords}

\noindent \begin{english}GPU Computing , Computational Geometry, Intersection Testing, Massive Parallelism, OpenCL, Ray, Tetrahedron, SIMD Parallelism\end{english} 

